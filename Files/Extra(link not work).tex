\chapter{Bernoulli Numbers}
\renewcommand\chapterillustration{Bernoulli}

\begin{defn}\label{defb}
Setting $B_0=1$ Bernoulli numbers\footnote{By convention if $B_1=\frac{-1}{2}$, the given Bernoulli sequence is called \textbf{first Bernoulli numbers} and \textbf{second Bernoulli numbers} if $B_1=\frac{1}{2}$.} are defined

\begin{align}
\sum_{j=0}^m\binom{m+1}{j}B_j=0
\end{align}

\end{defn}
One can also define the Bernoulli numbers\footnote{Discovered by Jacob Bernoulli($1654-1705$) and discussed by him in a posthumous work \textit{Ars Conjectandi}(1713).} using Pascal's triangle
\begin{defn}\label{defb2}

\begin{align*}
B_0&=1\\
B_2+2B_1+1&=B_2\\
B_3+3B_2+  3B_1+1&=B_3\\
B_4  +4B_3+6B_2+  4B_1+1&=B_4\\
B_5+5B_4  +10B_3+10B_2+ 5B_1+1&=B_5\\
B_6+ 6B_5+ 15B_4+ 20B_3+ 15B_2+ 6B_1+1&=B_6\\
B_7+7B_6+21B_5+35B_4+35B_3+21B_2+7B_1+ 1&=B_7\\
&\vdots
\end{align*}

\end{defn}

\begin{defn}
Pascal's triangle is a  triangular array of the  binomial coefficients. It is named after the French mathematician Blaise Pascal. Pascal's triangle determines the coefficients which arise in  binomial expansions. \\
For an example, consider the expansion
$$
(x+y)^7 = x^7+7x^6y+21x^5y^2+35x^4y^3+35x^3y^4+21x^2y^5+7xy^6+y^7
$$
Notice the coefficients are the numbers in row two of Pascal's triangle: 1, 7, 21, 35, 35, 21, 7, 1. In general, when a binomial like $x+y$ is raised to a positive integer power we have:
\begin{equation}
(x+y)^n=a_0 x^n+a_1 x^{n-1} y+a_2 x^{n-2} y^2+...+a_{n-1}xy^{n-1}+a_ny^n
\end{equation}
where the coefficients $a_i$ in this expansion are precisely the numbers on row n of Pascal's triangle.
\end{defn}
\subsection{Pascal Tree}
\centering
\begin{tikzpicture}[x=11.8mm,y=8mm]
  % some colors
  \colorlet{even}{cyan!60!black}
  \colorlet{odd}{orange!100!black}
  \colorlet{links}{red!70!black}
  \colorlet{back}{yellow!20!white}
  % some styles
  \tikzset{
    box/.style={
      minimum height=5mm,
      inner sep=.7mm,
      outer sep=0mm,
      text width=10mm,
      text centered,
      font=\small\bfseries\sffamily,
      text=#1!50!black,
      draw=#1,
      line width=.25mm,
      top color=#1!5,
      bottom color=#1!40,
      shading angle=0,
      rounded corners=2.3mm,
      drop shadow={fill=#1!40!gray,fill opacity=.8},
      rotate=0,
    },
    link/.style={-latex,links,line width=.3mm},
    plus/.style={text=links,font=\footnotesize\bfseries\sffamily},
  }
  % Pascal's triangle
  % row #0 => value is 1
  \node[box=odd] (p-0-0) at (0,0) {1};
  \foreach \row in {1,...,14} {
     % col #0 =&gt; value is 1
    \node[box=odd] (p-\row-0) at (-\row/2,-\row) {1};
    \pgfmathsetmacro{\value}{1};
    \foreach \col in {1,...,\row} {
      % iterative formula : val = precval * (row-col+1)/col
      % (+ 0.5 to bypass rounding errors)
      \pgfmathtruncatemacro{\value}{\value*((\row-\col+1)/\col)+0.5};
      \global\let\value=\value
      % position of each value
      \coordinate (pos) at (-\row/2+\col,-\row);
      % odd color for odd value and even color for even value
      \pgfmathtruncatemacro{\rest}{mod(\value,2)}
      \ifnum \rest=0
        \node[box=even] (p-\row-\col) at (pos) {\value};
      \else
        \node[box=odd] (p-\row-\col) at (pos) {\value};
      \fi
      % for arrows and plus sign
      \ifnum \col<\row
        \node[plus,above=0mm of p-\row-\col]{+};
        \pgfmathtruncatemacro{\prow}{\row-1}
        \pgfmathtruncatemacro{\pcol}{\col-1}
        \draw[link] (p-\prow-\pcol) -- (p-\row-\col);
        \draw[link] ( p-\prow-\col) -- (p-\row-\col);
      \fi
    }
  }
  \begin{pgfonlayer}{background}
    % filling and drawing with the same color to enlarge background
    \path[draw=back,fill=back,line width=5mm,rounded corners=2.5mm]
    (  p-0-0.north west) -- (  p-0-0.north east) --
    (p-14-14.north east) -- (p-14-14.south east) --
    ( p-14-0.south west) -- ( p-14-0.north west) --
    cycle;
  \end{pgfonlayer}
\end{tikzpicture}
\subsection{Pascal Identity}
\begin{align}\label{pascali}
\binom{n}{k}+\binom{n}{k-1}=\binom{n+1}{k}
\end{align}

\begin{proof}

\begin{align*}
\binom{n}{k}+\binom{n}{k-1}&=\frac{n!}{(n-k)!k!}+\frac{n!}{(n-(k-1))(n-(k-1))!(k-1)k!}\\
                           &=\frac{(n-(k-1))(k-1)n!+n!}{(n-(k-1) )(n-(k-1) )!(k-1)k!}\\
                           &=\frac{((n-(k-1))(k-1)+1)n!}{(n-(k-1) )(n-(k-1))!(k-1)k!}
                            =\frac{(n+1)!}{(n+1-k)!(k!)}
                            =\binom{n+1}{k}
\end{align*}

\end{proof}
\section{How to Compute $B_n$'s}
\begin{flushleft}
Using definition (\ref{defb2}) it is easy to compute that
\end{flushleft}
\begin{align}
B_0=1,\quad B_1=\frac{-1}{2},\quad B_2=\frac {1}{6},\quad B_3 = 0,\quad B_4 =\frac{-1}{30},\quad B_5 = 0,\quad B_6 =\frac{1}{42},\cdots
\end{align}

\section{Some Facts}
\begin{lem}[Binomial Convolution]\label{lembc}
Let $f(z)=\sum_{n\ge0}\frac{a_n}{n!}z^n, g(z)=\sum_{n\ge0}\frac{b_n}{n!}z^n$ and $h(z)=f(z)*g(z)$.
Then there exists $d_n$ such that
\begin{align}
h(z)=\sum_{n\ge0}\frac{d_n}{n!}z^n,\qquad \text{ with } d_n=\sum_{k=0}^{n}\binom{n}{k}a_{k}b_{n-k}
\end{align}

\begin{proof}
Multiplying $f(z)$ and $g(z)$ gives us
\begin{align*}
\biggl(\frac{a_0}{0!}z^0+\frac{a_1}{1!}z^1+\frac{a_2}{2!}z^2+\cdots\biggl)\biggl(\frac{b_0}{0!}z^0+\frac{b_1}{1!}z^1+\frac{b_2}{2!}z^2+\cdots\biggl)
\end{align*}
After expanding and regrouping we would get
\begin{align}\label{two}
\frac{a_0b_0}{0!0!}z^0+\biggl(\frac{a_0b_1}{0!1!}+\frac{a_1b_0}{1!0!}\biggl)z^1+\biggl(\frac{a_0b_2}{0!2!}+\cdots+\frac{a_2b_0}{2!0!}\biggl)z^2+\cdots
\end{align}
If we let $c_n$ to be the coefficient of $z^n$. For example $c_0=\frac{a_0b_0}{0!0!}$. Then we have the following formula
\begin{align}\label{three}
c_n=\sum_{k=0}^{n}\frac{a_kb_{n-k}}{k!(n-k)!}
\end{align}
Using (\ref{two}) and (\ref{three}) we write $h(z)$ as the following sum
\begin{align}\label{five}
h(z)=\sum_{n\ge0}c_nz^n
\end{align}
Now let's define a value $d_n$ such that
\begin{align*}
d_n&=n!c_n=n!\sum_{k=0}^{n}\frac{a_kb_{n-k}}{k!(n-k)!}\\
&=\sum_{k=0}^{n}\frac{n!a_kb_{n-k}}{k!(n-k)!}\\
&=\sum_{k=0}^{n}\binom{n}{k}a_kb_{n-k}
\end{align*}
This gives us that
\begin{align}\label{six}
\frac{d_n}{n!}=c_n
\end{align}
Substituting (\ref{six}) on (\ref{five}) completes the proof.
\begin{align*}
h(z)=\sum_{n\ge0}\frac{d_n}{n!}z^n,\qquad \text{ with } d_n=\sum_{k=0}^{n}\binom{n}{k}a_{k}b_{n-k}
\end{align*}

\end{proof}

\end{lem}

\section{Generating function}

\begin{thm}[Generating function for Bernoulli Numbers]\label{thmg}

\begin{align*}
\frac{z}{e^z-1} \text{ is a generating function for Bernoulli numbers.}
\end{align*}
\end{thm}

\begin{proof}
Let
\begin{align}
G(z)=\sum_{n\ge0}\frac{B_n}{n!}z^n,\qquad \text{ where } B_n \text{ stands for }n^{th} \text{ Bernoulli number.}
\end{align}
Multiply both sides by $e^z$

\begin{align}
e^zG(z)&=\sum_{n\ge0}\frac{z^n}{n!}\cdot\sum_{n\ge0}\frac{B_n}{n!}z^n\\
&=\sum_{n\ge0}\biggl( \sum_{k=0}^{n}\binom{n}{k}B_k\biggl)\frac{z^n}{n!}\qquad \text{ by lemma } (\ref{lembc}).\label{res}
\end{align}
Now, by our definition of the Bernoulli number in (\ref{defb})

\begin{align*}
\sum_{j=0}^m\binom{m+1}{j}B_j=0\qquad \text{ with } B_0=1.
\end{align*}
If we set $n=m+1$ and add $B_n$ to both sides, then we have
\begin{align}
\sum_{j=0}^{n-1}\binom{n}{j}B_j+B_n=\sum_{j=0}^{n}\binom{n}{j}B_j=0+B_n=B_n
\end{align}
or $B_n+1$ in the case where $n=m+1=1$.\\
This enables us to simplify the result in (\ref{res}) to get
\begin{align}
e^zG(z)=z+\sum_{n\ge0}B_n\frac{z^n}{n!}=z+G(z)
\end{align}
N.B. The $z$ at the bottom comes from the fact that at $n=1$, our result is
$$(B_1+1)\frac{z^1}{1!}=\frac{B_1}{1!}z^1+z$$
If we subtract $G(z)$ from both sides, we get
\begin{align*}
e^zG(z)-G(z)=z\\
G(z)(e^z-1)=z
\end{align*}
Dividing both sides by $e^z-1$ gives us
\begin{align}
G(z)=\frac{z}{e^z-1}
\end{align}

\end{proof}
\section{Why the  odds vanish?}
\begin{cor}
$B_{2i+1}=0$ if $i\ge1$.
\begin{proof}
From theorem (\ref{thmg}) we have
\begin{align*}
\frac{z}{e^z-1}=\sum_{n\ge0}B_n\frac{z^n}{n!}
\end{align*}
Then using the fact that $B_1=-1/2$, we have the following
\begin{align}\label{iden}
\sum_{\substack{n\ge1\\{n\neq 1}}}B_n\frac{z^n}{n!}=\frac{z}{e^z-1}-\frac{z}{2}
\end{align}
Now, look closely at the following algebraic simplification
\begin{align*}
\frac{z}{e^z-1}-\frac{z}{2}&=\frac{2z+z(e^z-1)}{2(e^z-1)}\\
&=\biggl(\frac{z}{2}\biggl)\frac{(e^z+1)}{(e^z-1)}\\
&=\biggl(\frac{z}{2}\biggl)\frac{(e^z+1)}{(e^z-1)}*\biggl(\frac{e^{-z/2}}{e^{-z/2}}\biggl)\\
&=\biggl(\frac{z}{2}\biggl)\biggl(\frac{e^{z/2}+e^{-z/2}}{e^{z/2}-e^{-z/2}}\biggl)
\end{align*}
If we replace $z$ by $-z$ in our final result, the identity remain unchanged($\frac{z}{e^z-1}-\frac{z}{2}$ is even function).
\begin{align*}
\biggl(\frac{-z}{2}\biggl)\biggl(\frac{e^{-z/2}+e^{-(-z/2)}}{e^{-z/2}-e^{-(-z/2)}}\biggl)=\biggl(\frac{-z}{2}\biggl)
\biggl(\frac{e^{-z/2}+e^{z/2}}{e^{-z/2}-e^{z/2}}\biggl)=\biggl(\frac{z}{2}\biggl)\biggl(\frac{e^{z/2}+e^{-z/2}}{e^{z/2}
-e^{-z/2}}\biggl)
\end{align*}
But this means that the same must hold true for $ \sum_{\substack{n\ge1\\{n\neq 1}}}B_n\frac{z^n}{n!}$. Since in (\ref{iden}), we showed that they are equal functions. So that we have

\begin{align*}
\sum_{\substack{n\ge1\\{n\neq 1}}}B_n\frac{z^n}{n!}=\sum_{\substack{n\ge1\\{n\neq 1}}}B_n\frac{(-z)^n}{n!}=\sum_{\substack{n\ge1\\{n\neq 1}}}(-1)^nB_n\frac{z^n}{n!}
\end{align*}
Matching up each of the terms according to powers of $z^n$, this gives us
\begin{align}
B_n=(-1)^nB_n
\end{align}
Thus, $B_n=0$, whenever $n$ is odd. Since we have excluded the case $n=1$, we conclude that
\begin{align*}
B_{2i+1}=0 \text{ if } i\ge1.
\end{align*}

\end{proof}

\end{cor}
\section{Simple but Elegant Application}
\subsection{Faulhaber's Formula}

\begin{align}
\sum_{k=0}^{m-1}k^n=\frac{1}{n+1}\sum_{k=0}^{n}\binom{n+1}{k}B_k m^{n+1-k}
\end{align}

\chapter{Zeta function}
\renewcommand\chapterillustration{Riemann}

\begin{defn} The zeta function is defined
\[
\zeta (s)=\sum_{n=0}^\infty \frac {1}{n^s}
\]
for each real values of $s$ with $s>1$.
\end{defn}
\begin{flushleft}
  Euler consider $s \in \mathbb{Z} $ and Riemann extend it to $s\in \mathbb{C}.$
\end{flushleft}
\section{The sum of reciprocal of prime diverges}

\begin{flushleft}
  In order to prove that the sum of reciprocal of primes diverge, we need to show the following holds.
\end{flushleft}
\begin{lem}\label{primediv0}
The power series(\ref{primedivpower}) of $f(x)$ which is
\[
f(x)=\ln{\biggl(\frac{1}{1-x}\biggl)}
\]
is given by
\[
\sum_{n=1}^{\infty}\frac{x^n}{n}
\]
\end{lem}
\begin{proof}
$f(x)=\ln{\biggl(\frac{1}{1-x}\biggl)}$, the Maclaurine series of $f(x)$ is by definition(\ref{maclaurine})
\[
f(x)=\sum_{n=0}^{\infty}\frac{f^{(n)}(0)x^n}{n!}
\]
Now,
\begin{align*}
f^{(1)}(x)&=\frac{1}{1-x}           &\Rightarrow f^{(1)}(0)&=1\\
f^{(2)}(x)&=\frac{1}{(1-x)^2}       &\Rightarrow f^{(2)}(0)&=1\\
f^{(3)}(x)&=\frac{2}{(1-x)^3}       &\Rightarrow f^{(3)}(0)&=2\\
f^{(4)}(x)&=\frac{2\cdot3}{(1-x)^4} &\Rightarrow f^{(4)}(0)&=2\cdot3\\
          &\vdots                   &                      &\vdots\\
f^{(n)}(x)&=\frac{(n-1)!}{(1-x)^n}  &\Rightarrow f^{(n)}(0)&=(n-1)!\\
\end{align*}
Hence,
\begin{align*}
f(x)=\sum_{n=0}^{\infty}\frac{f^{(n)}(0)x^n}{n!}=\sum_{n=0}^{\infty}\frac{(n-1)!x^n}{n!}=\sum_{n=0}^{\infty}\frac{x^n}{n}
\end{align*}

\end{proof}
\begin{thm}
The sum of reciprocal of primes
\[
\sum_{p}p\to \infty
\]
diverges(converges to $\infty$).
\end{thm}
\begin{proof}
Because of the fact that $\lim_{s\rightarrow 1^+}\zeta(s)=\infty$, it follows that
\begin{align}\label{primediv1}
\lim_{s\rightarrow 1^+}\ln(\zeta(s))=\infty
\end{align}
But,
\begin{align*}
\ln(\zeta(s))&=\ln\biggl(\sum_{n=1}^{\infty}\frac{1}{n^s}\biggl)\\
             &=\ln\biggl(\prod_{p}\frac{1}{1-1/p^s}\biggl)=\sum_{p}\ln\biggl(\frac{1}{1-1/p^s}\biggl)\\
             &=\sum_{p}\sum_{n=1}^{\infty}\frac{1}{np^{sn}}\tag{lemma (\ref{primediv0})}\\
             &=\sum_{p}\biggl(\frac{1}{p^s}+\sum_{n=2}^{\infty}\frac{1}{np^{sn}}\biggl)
\end{align*}
Hence,
\[
\ln(\zeta(s))=\sum_{p}\frac{1}{p^s}+\sum_{p}\sum_{n=2}^{\infty}\frac{1}{np^{sn}}
\]
Thus,
\[
\ln(\zeta(s))<\sum_{p}\frac{1}{p^s}+\sum_{p}\sum_{n=2}^{\infty}\frac{1}{p^{sn}}
\]
From geometric sum, we have
\[
\sum_{n=2}^{\infty}\frac{1}{p^{sn}}=\frac{1}{p^{2s}-p^s}
\]
Therefore, for $s>1$, we obtain that
\begin{align*}
\ln(\zeta(s))&<\sum_{p}\frac{1}{p^s}+\sum_{p}\frac{1}{p^{2s}-p^s}\\
             &=\sum_{p}\frac{1}{p^s}+\sum_{p}\frac{1}{p^{s}(p^s-1)}\\
             &<\sum_{p}\frac{1}{p^s}+\sum_{n=2}^{\infty}\frac{1}{n^{s}(n^s-1)}\\
             &<\sum_{p}\frac{1}{p^s}+\sum_{n=2}^{\infty}\frac{1}{n(n-1)}\\
             &<\sum_{p}\frac{1}{p^s}+\sum_{n=2}^{\infty}\frac{1}{(n-1)^2}
\end{align*}

Thus, it yields
\[
\lim_{s\rightarrow 1^+}\ln(\zeta(s))\leq \lim_{s\rightarrow 1^+}\biggl(\sum_{p}\frac{1}{p^s} +\sum_{n=2}^{\infty}\frac{1}{(n-1)^2}\biggl)
\]
and thus, by (\ref{primediv1}), it is evident that
\[
\lim_{s\rightarrow 1^+}\biggl(\sum_{p}\frac{1}{p^s} +\sum_{n=2}^{\infty}\frac{1}{(n-1)^2}\biggl)=\infty
\]
However, the series
\[
\sum_{n=2}^{\infty}\frac{1}{(n-1)^2}
\]
converges to a real number. Hence,
$$
\lim_{s\rightarrow 1^+}\sum_{p}\frac{1}{p^s}=\infty
$$
Therefore,
$$ \sum_{p}\frac{1}{p}=\infty $$
\end{proof}